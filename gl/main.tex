\input{../header}

\mode<beamer>{\usetheme{AnnArbor}}
\mode<beamer>{\setbeamertemplate{footline}}
\mode<beamer>{\setbeamertemplate{footline}[frame number]}
\mode<beamer>{\setbeamertemplate{frametitle continuation}[from second][\insertcontinuationcountroman]}
\mode<beamer>{\setbeamertemplate{navigation symbols}{}}

\mode<handout>{\pgfpagesuselayout{2 on 1}[letterpaper,border shrink=5mm]}

\def\CHAPTER{12}
\title{STATS 810 class \CHAPTER:\\Introduction to parallel statistical computing in R on greatlakes}
\author{Edward L. Ionides}

\setbeamertemplate{footline}[frame number]




\begin{document}

\maketitle

\mode<article>{\tableofcontents}

\mode<presentation>{
  \begin{frame}{Outline}
    \tableofcontents
  \end{frame}
}

\section{Logging in to greatlakes}

\begin{frame}{Requirements}

We follow \link{https://arc-ts.umich.edu/greatlakes/user-guide/\#document-2}{Section 1.2} of the \link{https://arc-ts.umich.edu/greatlakes/user-guide/}{greatlakes user guide}. As preliminaries, you need:

\begin{itemize}
  \item A Slurm account. You should already have a primary account, \code{stats\_dept1}, and a smaller backup account for if you exhaust your resourses, \code{stats\_dept2}. 
   \item A greatlakes cluster login account. If you have not yet filled in the form at \url{https://arc-ts.umich.edu/greatlakes/user-guide/} then do so.
   \item A umich internet address. Use the umich VPN if you are not on campus.
  \end{itemize}
\end{frame}


\begin{frame}{Connecting to greatlakes with macOS or Linux}
  \begin{enumerate}
  \item Open a Terminal window (recall that, on a Mac, this can be done using Control-Spacebar and typing Terminal) and type
\begin{knitrout}\small
\definecolor{shadecolor}{rgb}{0.969, 0.969, 0.969}\color{fgcolor}\begin{kframe}
\begin{alltt}
ssh uniqname@greatlakes.arc-ts.umich.edu
\end{alltt}
\end{kframe}
\end{knitrout}
where \code{uniqname} is your uniqname.
\item Login with your Kerberos level-1 password, and Duo two-factor authentication.
  \end{enumerate}
This creates a remote terminal shell on greatlakes.

\end{frame}



\begin{frame}{Connecting to greatlakes with Windows}
This is essentially the same as for macOS.
  \begin{enumerate}
  \item Follow instructions to install PuTTY at \url{https://documentation.its.umich.edu/node/350}
  \item Launch PuTTY and enter \code{greatlakes.arc-ts.umich.edu} as the host name, then click open.
If you receive a ``PuTTY Security Alert'' pop-up, this is completely normal, click the ``Yes'' option. This will tell PuTTY to trust the host the next time you want to connect to it. From there, a terminal window will open; you will be required to enter your UMICH uniqname and then your Kerberos level-1 password in order to log in. Please note that as you type your password, nothing you type will appear on the screen; this is completely normal. Press ``Enter/Return'' key once you are done typing your password.
\item Complete the request for Duo two-factor authentication.
  \end{enumerate}
This creates a remote terminal shell on greatlakes.

\end{frame}


\section{Moving files on and off greatlakes}

\begin{frame}[fragile]{Moving files on and off greatlakes: \code{scp}}

On Mac or Linux, you can use \code{scp} which has similar syntax to \code{cp}.

To copy \code{myfile} on your laptop to a subdirectory \code{mydir} of your home directory on greatlakes:
\begin{knitrout}\small
\definecolor{shadecolor}{rgb}{0.969, 0.969, 0.969}\color{fgcolor}\begin{kframe}
\begin{alltt}
scp myfile uniqname@greatlakes-xfer.arc-ts.umich.edu:mydir
\end{alltt}
\end{kframe}
\end{knitrout}
To copy an entire directory, use the \code{-r} flag for recursive copy:
\begin{knitrout}\small
\definecolor{shadecolor}{rgb}{0.969, 0.969, 0.969}\color{fgcolor}\begin{kframe}
\begin{alltt}
scp -r mydir uniqname@greatlakes-xfer.arc-ts.umich.edu:
\end{alltt}
\end{kframe}
\end{knitrout}
These commands can also be reversed to copy files from greatlakes to your machine. The following copies \code{mydir} back to the current working directory:
\begin{knitrout}\small
\definecolor{shadecolor}{rgb}{0.969, 0.969, 0.969}\color{fgcolor}\begin{kframe}
\begin{alltt}
scp -r uniqname@greatlakes-xfer.arc-ts.umich.edu:mydir .
\end{alltt}
\end{kframe}
\end{knitrout}
You will need to authenticate via Duo to complete the file transfer.
On Windows, you can use WinSCP or FileZilla. 
\end{frame}

\section{Working with batch jobs}

\begin{frame}{Cluster batch workflow}

\begin{enumerate}
\item You create a batch script and submit it as a job
\item Your job is scheduled, and it enters the queue
\item When its turn arrives, your job will execute the batch script
\item Your script has access to all applications and data
\item When your script completes, anything it sent to standard output
and error are saved in files stored in your submission directory
\item You can ask that email be sent to you when your jobs starts, ends,
or fails
\item You can check on the status of your job at any time,
or delete it if it's not doing what you want
\item A short time after your job completes, it disappears
\end{enumerate}
\end{frame}

\begin{frame}[fragile]{Useful batch commands}

Submit a job
\begin{knitrout}\small
\definecolor{shadecolor}{rgb}{0.969, 0.969, 0.969}\color{fgcolor}\begin{kframe}
\begin{alltt}
sbatch sample.sbat
\end{alltt}
\end{kframe}
\end{knitrout}

Query job status
\begin{knitrout}\small
\definecolor{shadecolor}{rgb}{0.969, 0.969, 0.969}\color{fgcolor}\begin{kframe}
\begin{alltt}
squeue -j jobid
squeue -u uniqname
\end{alltt}
\end{kframe}
\end{knitrout}
Delete a job
\begin{knitrout}\small
\definecolor{shadecolor}{rgb}{0.969, 0.969, 0.969}\color{fgcolor}\begin{kframe}
\begin{alltt}
scancel jobid
\end{alltt}
\end{kframe}
\end{knitrout}
Check a job script and estimate its start time
\begin{knitrout}\small
\definecolor{shadecolor}{rgb}{0.969, 0.969, 0.969}\color{fgcolor}\begin{kframe}
\begin{alltt}
sbatch --test-only sample.sbat
\end{alltt}
\end{kframe}
\end{knitrout}
\end{frame}

\begin{frame}{More Slurm commands to try}

\begin{tabular}{ll}
\code{sacct -u user} \hspace{5mm} & show recent job history
\\
\code{seff jobid} & show cpu utilization for jobid
\end{tabular}

\end{frame}

\section{R modules on greatlakes}

\begin{frame}[fragile]{R modules on greatlakes}

Sotware on greatlakes is packaged in modules which must be loaded
\begin{knitrout}\small
\definecolor{shadecolor}{rgb}{0.969, 0.969, 0.969}\color{fgcolor}\begin{kframe}
\begin{alltt}
module load R
\end{alltt}
\end{kframe}
\end{knitrout}
Other versions of R are available:
\begin{knitrout}\small
\definecolor{shadecolor}{rgb}{0.969, 0.969, 0.969}\color{fgcolor}\begin{kframe}
\begin{alltt}
module avail R
\end{alltt}
\end{kframe}
\end{knitrout}
We see that R3.6.1 is currently the default. For simple multicore computing, sending jobs to multiple cores on a single node, the default R module is appropriate. Other versions of R have been built and tested in other parallel environments, for example the \texttt{Rmpi} module runs R with mpi.

\end{frame}


\section{A test for foreach} 

\begin{frame}[fragile]{Set up test for foreach} 

\begin{itemize}
\item
The \texttt{gl} subdirectory of the \texttt{810f20} git repository has a file \texttt{test.sbat} which submits a batch job running the parallel \texttt{foreach} test in \texttt{test.R}.
\item
A basic Linux exercise is to set up a directory on greatlakes with these files, at which point you can run
\begin{knitrout}\small
\definecolor{shadecolor}{rgb}{0.969, 0.969, 0.969}\color{fgcolor}\begin{kframe}
\begin{alltt}
sbatch test.sbat
\end{alltt}
\end{kframe}
\end{knitrout}
to submit the job.
\item You can transfer the files from your laptop via \texttt{scp}, or by copy-paste, but it may be simplest to clone the class git repository into your greatlakes account,
\begin{knitrout}\small
\definecolor{shadecolor}{rgb}{0.969, 0.969, 0.969}\color{fgcolor}\begin{kframe}
\begin{alltt}
git clone https://github.com/ionides/810f20.git
\end{alltt}
\end{kframe}
\end{knitrout}
\end{itemize}

\end{frame}

\section{Editing text files on greatlakes}

\begin{frame}[fragile]{Editing text files on greatlakes}

\begin{itemize}
\item Inspect the text file \texttt{test.sbat}, for example by
\begin{knitrout}\small
\definecolor{shadecolor}{rgb}{0.969, 0.969, 0.969}\color{fgcolor}\begin{kframe}
\begin{alltt}
more test.sbat
\end{alltt}
\end{kframe}
\end{knitrout}
Is it fairly self-explanatory?
\item One thing that needs changing is to set your email address for alerts about jobs beginning and ending. 
\item To make these edits on greatlakes, you need a text editor.
\item It is convenient to use a text editor that runs in a terminal. Options include
\begin{knitrout}\small
\definecolor{shadecolor}{rgb}{0.969, 0.969, 0.969}\color{fgcolor}\begin{kframe}
\begin{alltt}
vi test.sbat
emacs -nw test.sbat
nano test.sbat
\end{alltt}
\end{kframe}
\end{knitrout}
\item It is useful to have some familiarity with each of these editors.
\end{itemize}

\end{frame}

\begin{frame}[fragile]{Comparing results}

\begin{itemize}
\item You are now ready to run a batch job
\begin{knitrout}\small
\definecolor{shadecolor}{rgb}{0.969, 0.969, 0.969}\color{fgcolor}\begin{kframe}
\begin{alltt}
sbatch test.sbat
\end{alltt}
\end{kframe}
\end{knitrout}
\item
From inspecting the code in \texttt{test.R}, we see that the results are saved in \texttt{test.csv}
\item
Compare the run times with the results from running this code on your laptop, as done in homework 11.
\item
Are you puzzled by this comparison? I was initially surprised. To shed some light on what is going on, try running the code in \texttt{test2.R} by
\begin{knitrout}\small
\definecolor{shadecolor}{rgb}{0.969, 0.969, 0.969}\color{fgcolor}\begin{kframe}
\begin{alltt}
sbatch test2.sbat
\end{alltt}
\end{kframe}
\end{knitrout}
What do you learn from comparing the output in \texttt{test2.csv} with \texttt{test.csv}?

\end{itemize}

\end{frame}

\section{Other ways to run R on greatlakes}

\begin{frame}[fragile]{Other ways to run R on greatlakes}

\begin{itemize}
\item It is sometimes useful to start an interactive session on greatlakes, particularly for debugging. This is done from the terminal as follows: 
\begin{knitrout}\small
\definecolor{shadecolor}{rgb}{0.969, 0.969, 0.969}\color{fgcolor}\begin{kframe}
\begin{alltt}
module load R
srun --nodes=1 --account=stats_dept1 --ntasks-per-node=8 \textbackslash{}
  --pty /bin/bash
\end{alltt}
\end{kframe}
\end{knitrout}
\item You can then run R in the terminal as usual, just by typing
\begin{knitrout}\small
\definecolor{shadecolor}{rgb}{0.969, 0.969, 0.969}\color{fgcolor}\begin{kframe}
\begin{alltt}
\hlstd{R}
\end{alltt}
\end{kframe}
\end{knitrout}
\item This R session will have access to the cores you have requested.
\item Here, we require \texttt{nodes=1} unless we use \texttt{Rmpi} since \texttt{library(doParallel)} alone cannot work with cores across different machines.
\item You can also run \link{https://arc-ts.umich.edu/greatlakes/user-guide/\#document-3}{web-based Rstudio}
However, batch jobs remain the basic tool for intensive statistical computing.
\end{itemize}

\end{frame}


\begin{frame}{Acknowledgments}
  \begin{itemize}
  \item
    This lesson is prepared for the STATS 810, Fall 2020.
  \item
    It builds on the \link{https://arc-ts.umich.edu/greatlakes/user-guide/}{Greal Lakes User Guide} and \link{https://ionides.github.io/531w20/cluster/STATS_531_Introduction_to_R_and_pomp_on_Great_Lakes.pdf}{notes by Charles Antonelli and John Thiels}.
\item
    Licensed under the \link{http://creativecommons.org/licenses/by-nc/4.0/}{Creative Commons Attribution-NonCommercial license}.
    Please share and remix non-commercially, mentioning its origin.
    \includegraphics[height=12pt]{../cc-by-nc}
  \item
    Compiled on \today.

  \end{itemize}

\end{frame}


\end{document}
